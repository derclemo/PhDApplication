Dear Professor Stadtfeld,\\

in July and August 2020, several countries launched probes to Mars because at that time the gravitational forces were aligned in a particularly energy-efficient way. 
Put differently, the gravitational forces were working for the probes, not against them.
Applying this reasoning to social systems we may ask whether specific ``\textit{windows of social momentum}'' can be identified? 
Instead of merely thinking about how to control a social network, we also wish to know \textit{when is the best time to control a social network?}\\

In my research, I want to uncover such pockets of controllability. 
Although this is certainly an ambitious claim, I am convinced that I have the skillset, motivation and endurance to achieve this. 
While studying for my BSc in Industrial Engineering and Management at Friedrich-Alexander-University Erlangen-Nürnberg (FAU), I became enthusiastic about computer science. 
This motivated me and two fellow students to develop an online platform called ``kowly'', enabling users to unearth inspiring content by sharing it with their community. 
I thereby gained not only a strong background in technology but also rigorous problem-solving and analytical skills.
After my bachelors, I pursued a master's degree in Computer Science at FAU with a focus on artificial intelligence and probabilistic modeling, which I completed in July 2020. \\

As part of my MSc, I have spent four months at the MIT Media Lab in Professor Alex `Sandy' Pentland's Human Dynamics group. 
It is there that I have learnt a lot about social networks and how they influence our lives in business, health, governance, and technology.
\red{I removed the sentence of financial systems for a few reasons. 
It is not needed. 
But more importantly, you do not want to evoke the impression that you are a `finance' guy. 
Yes, finance is one system that you will study during your PhD. 
But since he is a computational social scientist, he might want to distance himself (and by proxy, his students) from finance (probably not, but just playing it save). 
} 
While at MIT, I wrote my master thesis on the topic of ``Deep Learning-based Recommendation System for Trading Platforms''. 
\red{Add a reference to your MSc thesis.} 
Specifically, I have developed a strategy recommendation system on retail trading platform eToro (http://www.etoro.com).
One of eToro's distinguishing features is that users can observe the trades of other users and decide to automatically mimic all of their trades.
The Human Dynamics Lab has access to detailed information on all trades that have been executing on eToro over a period of three years. 
To develop a suitable recommendation system, we have embedded the eToro mimicking network (who follows whom). 
\red{I have removed some information here. Since sometimes, being a little more vague, yet not ambiguous, may be helpful. If you provide too much information, you might raise questions you don't want to answer in just 2 pages.} 
To this end, we have developed a graph embedding algorithm based on DeepWalk \cite{perozzi2014deepwalk}, but additionally also took into consideration network node properties (i.e. trader characteristics). 
Using the network embedding as input for downstream analysis tasks, we were able to profile and predict the users' trading preferences and behavior. 
Based on these insights, we demonstrated that our recommendation system is capable of providing each user with a list of either new relevant trading strategies or new follower-followee-relationships with similar taste.\\

From this time onwards, I was fascinated by the tools of computational social science to better organize society, government, and companies. 
But in order to pursue a career in this direction, I will need the help of collaborators and a strong team. 
Your group provides an exceptional setting consisting of outstanding academic expertise in combination with innovative research projects, giving me the opportunity to venture into my sphere of interest and enhance my personal as well as professional development.
One of your recent papers \cite{stadtfeld2020emergence} explains how social groups emerge and remain stable in various contexts.
Based on the idea that social groups interact via combination of forces of attraction and repulsion, the proposed model incorporates social mechanisms on the evolution of both positive and negative ties.
This further helps to describe why individuals refrain from developing ties to out-group members.
The results have shown that only the attraction and repulsion model generated stable structural groups, although the group features of the simulated network turned out to be too strongly stylized.
\red{Is the second half of the sentence, after ``although'' really needed?} \\

In view of the initial idea on examining ``\textit{windows of social momentum}'', I see clear synergies between ongoing research within your Social Networks Lab and my research interests. 
Your research on the stability of complex system sheds light on a topic identified during my time at MIT, which is towards a more sustainable design of complex systems.
Among other factors, stability is compromised by particularly dominant players in the underlying system \cite{lera2017prediction}.
So far, the formation of such dominance and associated destabilization was studied on the network itself. 
I propose that, instead, we study these issues by tracking the dynamics of the network in an embedding space. 
Concretely, I would like to analyze the co-movement of nodes (point clouds) in the embedding space. 
Following your idea of interacting forces in social systems, we can then look for constellations in which the point crowds are contracting (attracting force) or repelling (repulsing force). 
As a first use case, I would like to rely on the eToro data set to which I still have access. 
Indeed, essentially by construction, eToro is driven by the collective behavior of humans, which is why I think it will make for a valuable data set. 
Can we observe that successful traders tend to observe each other's strategies and ultimately end up all trading the same type of strategies (contraction)? 
Or is there pressure to diversify, in order to attract new mimickers (repulsion)? 
And how to disentangle these two orthogonal situations? 
Overall, this approach can help us better understand the complex workings of human interactions. 

In conclusion, being a doctoral student at ETH Zürich, a leading institution in the heart of Europe, is an unique opportunity to dedicate myself wholeheartedly as I am confident that this is my career's best turn to not only pursue my ultimate career goal but also satisfy my curiosity for understanding theories in computational social science. 
It has always been my age-long ambition to become a member of the computational social science community, and I am convinced that there is no better place to continue my academic career than the Social Networks Lab. 
Considering my academic performance so far and my desire to enrich mine and other's knowledge, I am convinced that I will be a valuable addition to your group, always striving to contribute the community in the best way that I can.\\

Thank you for considering my application.\\

Yours sincerely,\\
Tobias Clement\\\\\
